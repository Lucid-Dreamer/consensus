\documentclass[letterpaper,12pt]{article}
\setlength{\topmargin}{-1in}
\setlength{\textheight}{9.25in}
\setlength{\oddsidemargin}{.125in}
\setlength{\textwidth}{6in}

\setcounter{secnumdepth}{-1}

\usepackage{fontspec}
\usepackage{graphicx}

\setmainfont [Path = fonts/,
  UprightFont = Aller_Rg.ttf,
  BoldFont = Aller_Bd.ttf,
  ItalicFont = Aller_It.ttf
]{Aller}
\newfontfamily\titlefont [Path = fonts/,
  UprightFont = Kingthings_Printingkit.ttf,
]{Kingthings}

\newcommand{\TITLE}[1]{\begin{center}{\titlefont\Huge\textbf{#1}}\end{center}}
\newcommand{\SECTION}[1]{\vspace{.5em}{\noindent\titlefont\large\textbf{#1}}

}

\begin{document}

\TITLE{Consensus Important Concepts}

\SECTION{The Consensus}
The Consensus is a great unknown for many mages. Some say that it's
the collective beliefs about reality from all intelligent life forms,
some say it's a semi-conscious manifestation of the universe's
principals. (The same principals that mages regularly violate and
skirt around.) Whatever the true nature of The Consensus, it existence
means that reality isn't just torn asunder by people with fantastic
powers, like yourself.

\SECTION{Mage}
A Mage is someone who has the ability to alter reality as we know
it. They consider themselves awakened from the sleep of ignorance, and
are closer to the pinnacle of human achievement than most anyone
else.

\SECTION{Sleeper}
A Sleeper is a regular average person who does not have the ability to
rewrite reality with a thought. Magic has no place in their worldview,
or even if they believe in it, the fundamental nature of what Mages do
is completely alien to their brains. Some refer to them as the eyes
and ears of The Consensus.

\SECTION{Rending Magic}
Most magic that Mages do is barely distinguishable from coincidence or
extreme luck. There's a reason for this. That kind of magic is easy,
and safe. Instead of creating money out of thin air, a Mage prefers to
find the exact location buried treasure is hidden, right in their
backyard. Maybe it was there before they did their magic, and their
magic just allowed them to find it. Or maybe their magic put it there
just minutes before it was dug up. Either way, the casual onlooker
(and therefore the Consensus) can't tell the difference. This kind of
magic acts in harmony with reality, circumvents the rules of the
universe, and nobody's the wiser. There's a reason you don't see Mages
throwing balls of fire down the streets of Chicago. To do that
requires tearing a rather large hole in reality, and that's when
reality fights back. That's Rending magic, and it has
consequences. (If the magic can't be passed off as coincidence or
happenstance to the casual onlooker, the effect is likely Rending.)

\SECTION{Lookout}
The Consensus, while it seems to be all-powerful, is not quite
all-knowing. A Mage can do as much Rending magic as she likes when
there is no one around to see it. However, the moment that Rending
magic affects someone who doesn't share her exact views on the world,
their realities conflict, and there will be consequences. A lookout
is any sleeper, or any Mage who has an opposed paradigm. If the effect
doesn't make sense in the worldview of someone observing it or
otherwise affected by it (even tangentially), they are a lookout, and
the effect is Rending.

Note that this means Mages can be gods in their own little space,
doing whatever they want. But no matter what it is they do, it will
likely have an effect on a lookout sooner or later, especially if they
ever want to see another person again.

\SECTION{Discord}
When you rip holes in reality, it leaves a mark on you. Even when it
doesn't immediately hurt you, that can build up, and if you let it
build up too far, you're in for one heck of a whoopin' from reality.

\SECTION{Backlash}
This is the universe's way of fixing itself after a hole is torn in it
by Rending magic. It is also the punishment for Mages who break the
rules too egregiously. Sometimes a backlash is minor, the equivalent
of a slap on the wrist. Sometimes you're banished from reality
itself. Depends on how much you've been messing with reality, and how
much it feels like messing back.

\SECTION{Anchor}
When we say that Mages are the pinnacle of humanity, we mean it. But
that's not always a good thing, right? There's a line between the top
of humanity and something inhuman that is so easy for a Mage to step
over. Your Anchors are the things that keep you on this side of the
line. A loved one who knows nothing about magic, an estate bound up in
the history of your family, the stray cat you took in off the street,
or the picture of your brother who died the day you awakened. Anchors
are physical objects, or beings, that keep you grounded in the mortal
world. But they also hold you back from your fullest potential. Do you
want to shed the shackles of your Anchors in order to transcend, or
remain among them, being a little bit better, but know you will never
reach your full potential?

Anchors can not be magical in nature. Other Mages, spirits you've
summoned, these things aren't grounded enough in reality to keep you
there. A cat that happens to be your familiar, fine. A spirit familiar
you summoned that happens to take the shape of a cat, not so fine.

Mages start with either 3 or 4 Anchors, and after they start losing
them, can get more, but only ever back up to 3. You can take a new
Anchor instead of an advancement any time that you have 2 or fewer
remaining Anchors. Explain why they are now something that keeps you
grounded, keeps you human.

Some Anchors have agency. People and animals are the chief among
them. Keep in mind that these Anchors have the agency to protect
themselves from danger sometimes, but also have the agency to turn
their backs on you. If they do so, it's up to you at that point if
they remain an Anchor...but if they do, they won't make it easy on
you.

A Mage who has lost all their Anchors is adrift in a sea of magic, no
longer quite human, they have become something...else. When you lose
your last Anchor, you will get to dictate one scene that is how you
react to losing that Anchor, and how it sends you over the edge of
humanity. Then you hand your character sheet over to the GM, and start
a new character.

As an interesting note, it is easier to bring a Mage back from the
dead than it is to restore an Anchorless Mage to humanity.

\SECTION{Tethered}
A Mage with more than half of their original starting number of
Anchors is Tethered to reality. They are still fairly grounded, and
are unlikely to suddenly go for an endless sojourn in the vast seas of
infinity. However, their minds are also that much more clouded by
physical considerations, and they cannot reach their highest
potential.

A Tethered Mage cannot choose any Advancements from the third section,
and will have a harder time demonstrating true magical superiority.

\SECTION{Status Tracks}
Many things in Consensus have a sort of progression. Even if
individual events aren't precisely connected, one layers on top of
another, getting more and more intense until reaching a breaking
point.

To represent this, Consensus uses Status Codes similar to those used
in medical transport.

A set of status codes is as follows:

\begin{itemize}
  \setlength{\itemsep}{0em}
\item \textbf{Code 50} Basic transport (not serious)
\item \textbf{Code 40} Serious case (IV started)
\item \textbf{Code 30} Trauma case
\item \textbf{Code 20} Acute Trauma case
\item \textbf{Code 10} Critical Trauma case
\item \textbf{Code N} Newsworthy event
\end{itemize}

The exact meaning of each Code will depend on what the status track is
tied to, especially Code N, the Newsworthy event. Code N is always
something with major, usually irrepable, repercussions, at least in
the scope of the Status Track.

For most Status Tracks, Code 50 and Code 40 will recover on their own,
given time. Once a Code 30 has been reached, the track will slide
downwards without concentrated outside intervention.

Status Tracks are usually both prescriptive and descriptive. That is,
if the rules say to advance a Status Track by two codes, then do so,
and describe what happens. In addition, if something narratively
happens that would indicate a code on the Status Track, then advance
the Status Track to reflect the new state.

\SECTION{Harm}
Harm in Consensus uses a Status Track.

Every time you take Harm, it will be a certain number. Cross off one
Code, in order, for each level of Harm taken, and write down what
injury this represents, keeping in mind the severity each Code
indicates. So, 2 Harm to an uninjured character, delivered by a blow
from a ghost's cutlass, might represent a cut to the arm that hits a
big vein and starts to bleed badly (``deep cut'' for Code 50 and
``blood loss'' for Code 40). The same 2 Harm to a character who
already has injuries for Codes 50-30 likely indicates a paralyzed arm
and severed artery. In addition, when you take Harm, roll the Harm
move (it is possible to suffer minor injuries, which do not change
your Harm status track, but do trigger the Harm move, represented by
taking ``0 Harm'').

As with other Status Tracks, once Code 30 has been reached, the track
will get worse without help. For harm, this represents injuries that
will kill you without medical attention. A punctured lung, a cracked
skull, the types of things that if you just try to rest and recuperate
will end up killing you. You need medical attention, and soon.

Even before reaching Code 30, pushing yourself when you're injured in
any way can make things worse. Being strenuous with a cracked rib is
just crying out to end up with a punctured lung after all. (GMs take
note, and refer to the Adjust a Status Track GM move).

Use these amounts of Harm for rough guidelines.

\begin{itemize}
  \setlength{\itemsep}{0em}
\item \textbf{0 Harm} Small cuts
\item \textbf{1 Harm} A solid punch
\item \textbf{2 Harm} A knife wound
\item \textbf{3 Harm} A gunshot
\item \textbf{4 Harm} A grenade
\end{itemize}

Code N for Harm indicates that the Mage is about to die. If you take
Harm that causes you to reach Code N, including if you don't get
medical attention while below Code 40 for too long, you'll be rolling
the death move.

\SECTION{Death}
Sometimes things die. That includes Mages. When you die, you will get
one final scene to impart wisdom on your friends, or just unleash an
unholy blast of Rending magic at the one who killed you. Death is not
always the end, especially for Mages. But it is the end for now.

\SECTION{Magical Preparations}
Humans are fragile creatures, and Mages tend to be a careful lot, so
even if they don't think they're explicitly in danger, they'll have
prepared something to keep them safe ``just in case.'' This could be a
bit of luck magic to make sure they just happen to be out of danger's
way, or a burst of superhuman reflexes to move you out of the
way. Point is, most Mages can take a bit of harm without any permanent
damage. They don't often get hit by oncoming traffic just crossing the
road. But those defenses take some time to prepare, so once they're
gone, a Mage is just as vulnerable as any other person.

Above Code 50, a Mage has ``Preparations.'' When they take Harm and
they have not already expended their Preparations, all of the Harm is
blocked by the Preparations, as long as this can be explained in a
harmonious manner. Rending defensive magic is always active. Even with
Preparations, you still roll the Harm move as if you had taken 1 Harm.

At certain times, Harm may bypass Preparations, including if there is
no conceivable way the Mage could have avoided it (no amount of
passive preparations will protect you when you dive into a lake of
lava), or for many types of sacrifices.

Preparations can be regained by taking a small amount of time in a
safe location, reconnecting with one of your Anchors.

\SECTION{Paradigm}
Just as it is not enough to say that a plane flies ``because of
physics,'' a Mage knows better than that they can create wondrous
effects ``because of magic.'' A Mage's Paradigm is their personal
belief system that explains how magic works. For one Mage, it might be
that there is an ancient language, infused with the power to affect
the world if spoken aloud. Armed with this belief, the Mage is able to
cast spells that rival any fantasy wizard. (What the Consensus has to
say about this is another matter entirely.)

Another Mage might argue that they don't do magic at all (and may even
debate its existence in the universe). They could, however, have a
scientific theory of how the human mind can influence wave function
collapse, allowing someone with a complete enough understanding of a
technological device to do things that look impossible. This Mage's
paradigm is likely much more limited than the more traditional,
wizard-like one, but can create effects which are just as fantastic,
and may even be easier to disguise among mundane behaviors.

Paradigms are usually identified by an overarching descriptive belief,
a set of abilities a Mage carrying said belief is particularly capable
of, and effects which are completely outside of that belief system. A
Mage dedicated to science above all else likely can't justify faith
healing as a part of their conception of reality.

Of course, to be a Mage is to go beyond the limitations of the human
mind, and a Mage can always attempt to overcome their preconceived
notions and force an attempt to work through sheer force of will. The
Mage will be working against their own belief, though, and that is a
lot more personal than the pressure the Consensus normally exerts.

Rolling something aligned with your paradigm is done as normal, if you
attempt magic that isn't specifically aligned, but isn't opposed, you
roll at a -1. If you attempt magic that is opposed to your paradigm,
it is Rending, and you count as a Lookout to your own magic. (And you
are most certainly affected by it.)

\SECTION{Place of Power}
Places of power are locations with a high concentration of magic, for
one reason or another. These tend to be naturally occurring, like
ley-line nodes, which are jealously guarded secrets, or else
constructed with a copious amount of time and energy from many
Mages. Places with residual magic like this are paramount when putting
together rituals of great power, so they are coveted by groups of
Mages wishing to work large magics, or singular Mages who need to draw
on great amounts of power.

Some playbooks have access to a place of power through optional
moves. Other places of power may show up during play. Be careful
allowing the players to permanently gain access to a place of power if
there isn't a playbook move specifically granting it to them. There
should be a real benefit for taking that move, and that's easier
access to a space in which rituals work. It may be fine to give
players that work for it temporary access to a place of power, or even
access to a temporary place of power.

\SECTION{Consensus Enforcer}
Consensus Enforcers are concrete bits of the forces that cause
Backlash. When a Mage has been particularly troublesome to Reality,
instead of being marked with a Discordant Sign, a Consensus Enforcer
may manifest. These spirits have at least animal-like intelligence and
powers to cause trouble for the Mage or Mages whose actions lead to
the Backlash. They can usually be handled like other spirits, but grow
stronger in the presence of Discord, which can make overcoming them an
exercise in care and subtlety.

\SECTION{Reality Bubbles}
Sometimes, a Mage who has been bending Reality with their mind will
suddenly find Reality bending their mind, or their very being,
back. When this is the result of a suitably strong Backlash, the
effect can be persistent. There are three kinds of Reality
Bubbles. Each acts as its own type of prison for the Mage, which some
incredibly lucky Mages may be able to escape on their own, but most
will need the help of others, if they are able to escape at all.

The most drastic and obvious type of Reality Bubble is the Discordant
Realm. The Mage is ejected from reality, disappearing from wherever
they were when the Backlash struck, into a bubble of weird rules and
wild magic. If the Mage is able to survive the dangers the demiplane
and bring themselves back in line with the Consensus, they may be able
to return to this reality. In addition, the point of departure almost
always leaves a mark in the world, which other brave or foolhardy
Mages may be able to use to follow an ejected friend and seek to bring
them back. Discordant Realms are Reality's ultimate defense against
harm, akin to an oyster's pearl, and if a Mage is not careful, they
can remain stuck forever, drifting in the spaces between planes. The
fate of many Marauders is to leave reality for a Discordant Realm.

More subtle is a Reality Fracture. This kind of Reality Bubble leaves
the Mage physically in the real world, but they see an entirely
different reality, superimposed over this one and only mostly lining
up. If the Mage is able to pierce the illusion, through force of will,
luck, or with the help of allies, they may be able to escape the
Fracture and allow Reality, and their mind, to heal. If the Mage buys
into the Fracture too strongly, however, it may begin to diverge from
reality, eventually leaving and taking them with it (if they don't
follow a hallucination into now-hidden traffic or otherwise get
themselves killed due to ``mistranslations.'')

Most subtle of all is a type of Reality Bubble called Discordant
Visions. In this case, the Mage largely sees things as they are, but
with important differences, recurring hallucinations. Perhaps UFOs
hang in the sky, motionless, and ignored by all. The Mage may take it
upon themselves to inform everyone of the alien threat, or take the
fight to them. This kind of Reality Bubble allows the Mage to most
easily see and interact with their friends and allies as they really
are, which may be helpful, but if this leads the Mage to believe that
they are the only one who can understand the true way things are, and
everyone else, even their friends, are trying to undermine
them, it can still be extremely difficult to escape from.

A Mage with an exceptionally strong and careful will may be
able to overcome a Reality Bubble on their own, but most require
outside assistance. Even then, help can be hard to give, because most
Mages are used to their own minds being the one trustworthy source of
information they have. In addition, if the subjects of a Reality
Bubble would inspire the Mage to more Rending magic, they may push
themselves even deeper into the Bubble.

A Mage suffering from ongoing hallucinations in a Reality Bubble might
see gremlins threatening to make trouble wherever they go. The
hallucinations can sometimes be dispelled on an individual level by
mental focus and fortitude, but the overall Reality Bubble
remains. Escaping from it may require anything from steadfastly
ignoring the gremlins and not being goaded into using Rending magic by
them for long enough, to finding the (still hallucinated) nest they
are coming from and erasing it from their mind.

A Mage who is perceiving an entirely different reality as the result
of a Reality Fracture may wander around the world, or may become
catatonic, traveling an unreal world in their mind. In either case,
the world they are seeing could be anything from a fantastic
faerieland, interpreting passers-by as elves, to the world of a noir
film, to a school in ancient Greece.

A Mage who has given in to their Reality Bubble and believes it is
Reality is called a Marauder, and they are terrifying examples of how
Magehood can go wrong.

% A Mage who runs out of Anchors while in a Quiet probably always
% becomes a Marauder. We should write something about that,
% somewhere. - Eric

\SECTION{Discordant Sign}
One of the simplest outcomes of a Backlash, a Discordant Sign is a weird
manifestation that occurs as Reality patches itself back up. A
Discordant Sign can be just about anything that lasts more than a few
moments and hinders the Mage. A Mage might grow tiny devil horns that
don't fade for a couple days, or find that they cannot abide the smell
of color blue until they have crossed running water three times. More
subtly, a Mage's car might simply refuse to start until Reality has
become less twisted up around them. Discordant Signs are immensely
varied, but all have one thing in common: Trying to fix them with
magic only ever makes them worse.

\SECTION{Experience}
Whenever the rules tell you to, you mark experience. This comes from a
few different sources, including rolling a highlighted stat, resetting
secrets with another character, or the end-of-session move. Every time
you mark your fifth experience, erase them all and make an
advancement.

\SECTION{Advancement}
Whenever you mark your fifth experience, you reset down to 0 and make
an advancement. Your first 3 advancements must come from the “Basic
Advancements” category. Additional ones after that may come from
``Basic Advancements,'' or ``[Advanced Advancements].'' If you are not
Tethered, any time you would be able to choose from ``[Advanced
  Advancements],'' you may also choose from ``[Super Advancements].''
Each advancement option may be chosen only once, except for the ``Gain
a new Anchor” advancement, which may be taken any time you have fewer
than 3 Anchors remaining.

\SECTION{Highlighting}
At the beginning of each session, the GM and the player with the
highest secrets towards you each will choose one stat that they want
to see you do. Note these two stats down. Any time you roll one of
these stats, mark experience.

\SECTION{Playbook Moves}
Each playbook starts with 3 moves. Between 0 and 2 moves will be
marked ``You have this move by default.'' Choose from the rest of the
moves listed in that playbook until your total starting moves = 3.

\end{document}
