% -*- Mode: latex -*-
\documentclass[letterpaper,10pt]{article}

\usepackage{multicol}
\usepackage{geometry}
\usepackage{xltxtra}
\usepackage{color}
\usepackage{fontspec}
\usepackage[compact]{titlesec}
\usepackage{setspace}
\usepackage{tikz}
\usepackage{graphicx}
\usepackage{textpos}
\usepackage{lipsum}
\usepackage{fancyhdr}

\geometry{letterpaper, landscape}

\setlength{\columnseprule}{1pt}

\setmainfont [Path = fonts/,
  UprightFont = Aller_Rg.ttf,
  BoldFont = Aller_Bd.ttf,
  ItalicFont = Aller_It.ttf
]{Aller}
\newfontfamily\titlefont [Path = fonts/,
  UprightFont = Kingthings_Printingkit.ttf,
]{Kingthings}

\titlespacing{\section}{0pt}{*0}{*0}
\titlespacing{\subsection}{0pt}{*0}{*0}

\setcounter{secnumdepth}{-1}

\newenvironment{move}{}{}
\newcommand{\TRIGGER}[1]{\textbf{#1}}
\newcommand{\SEPARATOR}{\begin{center}\noindent\rule{6cm}{2pt}\end{center}}

\pagestyle{fancyplain}
\renewcommand{\headrulewidth}{0pt}
\setlength{\columnsep}{.3in}
\setlength{\oddsidemargin}{-.5in}
\setlength{\textwidth}{10in}
\setlength{\textheight}{10in}
\cfoot{}


\chead{\titlefont\Huge\textbf{The Basic Moves}}

\begin{document}
\begin{multicols}{3}

  \begin{move}
    When you \TRIGGER{assert your Paradigm over another Mage's}, roll
    +Charm (-1 if Tethered). On a 10+, hold 3, or on a 7-9, hold
    1. You can spend your hold 1 for 1 to:
    \begin{itemize}
      \setlength\itemsep{-.5em}
    \item Have them mark experience (if an NPC, they comply with a
      reasonable request)
    \item Give them a +1 or -1 on their next roll (if an NPC, enhance
      or counter their next use of magic)
    \item Cause their current Discord to backlash
    \end{itemize}
    On a miss, they hold 1 against you, on the same terms.
\begin{movedetail}
  For Mages, there isn't all that much hope of manipulating each other
  subtly. You can tell another mage what you want them to do, but when
  they also have reality-bending powers, there's rarely any way to
  guarantee they'll do what you want. That is, unless you're stronger
  than they are magically. That's where this move comes in. Mages, out
  of pragmatism, must operate on a hierarchical structure, which is to
  say that if another mage is magically superior to you, they have a
  measure of control over you. If you demonstrate your superiority
  over another mage, you're also showing them some of how you view the
  world, they understand you slightly better (or maybe slightly worse)
% Maybe rework this to include bits about leveraging your magical
% superiority into getting others to do what you want.
\end{movedetail}
  \end{move}

  \SEPARATOR

  \begin{move}
    When you \TRIGGER{impress a non-Mage magical being with a feat of
      magic in order to sway them}, tell them what you want. They will
    ask for a promise in return. Roll +Charm (-1 if Tethered). On a
    10+, they will do what you want if you make the promise. On a 7-9
    they may choose to:
    \begin{itemize}
    \item Give you something else they think you want or need
    \item Make themselves scarce
    \item Require you to fulfill the promise before they help
    \end{itemize}
  \end{move}

  \SEPARATOR

  \begin{move}
    When you \TRIGGER{use a sleeper} for your own ends, you must first
    present something they want. Roll +Charm. On a 10+, they will do
    what you want to the best of their abilities, but on a 7-9,
    they'll need proof that they'll get what they want, or they'll
    need it before they do anything for you.
  \end{move}

  \columnbreak
    
  \begin{move}
    When you \TRIGGER{sell a lie} (to a person or to the Consensus),
    explain why they might believe it and roll +Wits. On a 10+, you've
    done it, with no-one the wiser. On a 7-9 you only mostly get away
    with it; the GM will offer you a hard bargain, worse outcome, or
    tough choice.
\begin{movedetail}
  This is one of the most complex moves in the game, and also one of
  the most versatile. Any time you're trying to slip something by a
  person in conversation, or otherwise trick them, this is the move to
  go to. Any time you try to slip a subtle, harmonious magical
  effect past the filters of reality, this is the move to go
  to. Pretty much any magic that doesn't fit into one of the other
  moves probably can fit into this one.
\end{movedetail}
  \end{move}

  \vspace{1em}
  
  \SEPARATOR

  \begin{move}
    When you \TRIGGER{outmaneuver, outpace, or outfox}, roll
    +Grace. On a 10+, you're scot free. On a 7-9, choose 2, or let
    the GM choose the worst 1:
    \begin{itemize}
      \setlength\itemsep{0em}
    \item You stumble, trip up, or falter
    \item You attract attention
    \item You have to leave something behind
    \item You leave an obvious trail
    \item You damage something
    \end{itemize}
\begin{movedetail}
  This move is particularly useful for when you're trying to do
  something quickly, or craftily. The GM might call for this move when
  you're trying to outpace a rogue mage in a foot-race, or when you're
  trying to avoid automagical countermeasures on that machine you just
  jacked yourself into. This move will save your bacon a hundred times
  over before you're done with it.
\end{movedetail}
  \end{move}

  \SEPARATOR

  \begin{move}
    When you \TRIGGER{keep your head down} to avoid attention, roll
    +Wits. On a 10+, you avoid notice, suspicion, or detection. On a
    7-9, someone, but not everyone, notices you. Choose two:
    \begin{itemize}
      \setlength\itemsep{0em}
    \item They keep quiet about it
    \item They weren't actively looking for you
    \item They don't start to follow you
    \item They don't have backup
    \end{itemize}
  \end{move}

  \SEPARATOR

  \begin{move}
    When you \TRIGGER{do Rending magic} and it affects a Lookout, gain
    2 Discord, Backlash, and roll +nothing (the Consensus doesn't care
    how smart or sexy you are).

    On a 10+ choose up to 3, on a 7-9 choose exactly 2.
    \begin{itemize}
      \setlength\itemsep{0em}
    \item You only gain 1 Discord
    \item You don't Backlash
    \item You don't cause collateral damage
    \item You don't add a Restriction to your Paradigm
    \end{itemize}
    On a miss, choose one, but not ``You don't Backlash.''
  \end{move}

  \columnbreak
  
  \begin{move}
    When you \TRIGGER{exert yourself} physically on the world or another
    person, roll +Body. On a 10+, choose 3. On a 7-9, choose 2:
    \begin{itemize}
      \setlength\itemsep{0em}
    \item You accomplish your goal
    \item You aren't injured in the process
    \item You don't cause any collateral damage
    \item You impress or dismay the target or onlookers
    \end{itemize}
  \end{move}

  \SEPARATOR

  \begin{move}
    When you \TRIGGER{Backlash}, roll +Discord, then reset Discord to
    0.

    On a 10+ choose 1:
    \begin{itemize}
      \setlength\itemsep{0em}
    \item You begin to see Discordant Visions
    \item You enter a Reality Fracture
    \item You are sucked into a Discordant Realm
    \end{itemize}

    On a 7-9, choose 2:
    \begin{itemize}
      \setlength\itemsep{0em}
    \item You take 3 Harm
    \item You don't reset your Discord to 0
    \item A Discordant Sign manifests
    \item A Consensus Enforcer manifests
    \end{itemize}

    On a miss, you shake it off with minimal lasting effects.
  \end{move}

  \SEPARATOR

  \begin{move}
    When you \TRIGGER{do magic that is Aligned to your Paradigm}, take
    +1 forwards when following up on that magic.
  \end{move}

  \SEPARATOR

  \begin{move}
    When you \TRIGGER{help or hinder} another PC, roll +Anchors. On a
    10+ they get a +1 or -2 to their roll, your choice. On a 7-9 they
    still get the bonus or penalty, but you open yourself to cost,
    danger, or retribution.
  \end{move}

\end{multicols}
\end{document}
