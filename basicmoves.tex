% -*- Mode: latex -*-
\documentclass[letterpaper,10pt]{article}

\usepackage{multicol}
\usepackage{geometry}
\usepackage{xltxtra}
\usepackage{color}
\usepackage{fontspec}
\usepackage[compact]{titlesec}
\usepackage{setspace}
\usepackage{tikz}
\usepackage{graphicx}
\usepackage{textpos}
\usepackage{lipsum}
\usepackage{fancyhdr}

\geometry{letterpaper, landscape}

\setlength{\columnseprule}{1pt}

\setmainfont [Path = fonts/,
  UprightFont = Aller_Rg.ttf,
  BoldFont = Aller_Bd.ttf,
  ItalicFont = Aller_It.ttf
]{Aller}
\newfontfamily\titlefont [Path = fonts/,
  UprightFont = Kingthings_Printingkit.ttf,
]{Kingthings}

\titlespacing{\section}{0pt}{*0}{*0}
\titlespacing{\subsection}{0pt}{*0}{*0}

\setcounter{secnumdepth}{-1}

\newenvironment{move}{}{}
\newcommand{\TRIGGER}[1]{\textbf{#1}}
\newcommand{\SEPARATOR}{\begin{center}\noindent\rule{6cm}{2pt}\end{center}}

\pagestyle{fancyplain}
\renewcommand{\headrulewidth}{0pt}
\setlength{\columnsep}{.3in}
\setlength{\oddsidemargin}{-.5in}
\setlength{\textwidth}{10in}
\setlength{\textheight}{10in}
\cfoot{}


\chead{\titlefont\Huge\textbf{The Basic Moves}}

\begin{document}
\begin{multicols}{3}

  \begin{move}
    When you \TRIGGER{assert your Paradigm over another Mage's}, roll
    +Charm (-1 if Tethered). On a 10+, hold 3, or on a 7-9, hold
    1. You can spend your hold 1 for 1 to:
    \begin{itemize}
      \setlength\itemsep{-.5em}
    \item Have them mark experience (if an NPC, they comply with a
      reasonable request)
    \item Give them a +1 or -1 on their next roll (if an NPC, enhance
      or counter their next use of magic)
    \item Cause their current Discord to backlash
    \end{itemize}
    On a miss, they hold 1 against you, on the same terms.
\begin{movedetail}
  For Mages, there isn't all that much hope of manipulating each other
  subtly. You can tell another mage what you want them to do, but when
  they also have reality-bending powers, there's rarely any way to
  guarantee they'll do what you want. That is, unless you're stronger
  than they are magically. That's where this move comes in. Mages, out
  of pragmatism, must operate on a hierarchical structure, which is to
  say that if another mage is magically superior to you, they have a
  measure of control over you. If you demonstrate your superiority
  over another mage, you're also showing them some of how you view the
  world, they understand you slightly better (or maybe slightly worse)
% Maybe rework this to include bits about leveraging your magical
% superiority into getting others to do what you want.
\end{movedetail}
  \end{move}

  \SEPARATOR

  \begin{move}
    When you \TRIGGER{impress a non-Mage magical being with a feat of
      magic in order to sway them}, they will expect something in
    return. Roll +Charm (-1 if Tethered). On a 10+, they will do what
    you want if you promise to do it. On a 7-9 they may choose to
    instead:
    \begin{itemize}
    \item Give you something else they think you want or need
    \item Make themselves scarce
    \item Require you to fulfill the promise before they help
    \end{itemize}
  \end{move}

  \SEPARATOR

  \begin{move}
    When you \TRIGGER{use a sleeper} for your own ends, you must first
    present something they want. Roll +Charm. On a 10+, they will do
    what you want to the best of their abilities, but on a 7-9,
    they'll need proof that they'll get what they want, or they'll
    need it before they do anything for you.
  \end{move}

  \SEPARATOR

  \begin{move}
    When you \TRIGGER{do magic that is Aligned to your Paradigm} and
    roll a 12+, take narrative control briefly to describe how it
    works above and beyond your expectations, and how that goes well
    for you.
  \end{move}

  \columnbreak
    
  \begin{move}
    When you \TRIGGER{sell a lie} (to a person or to the Consensus),
    explain why they might believe it and roll +Wits. On a 10+, you've
    done it, with no-one the wiser. On a 7-9 you only mostly get away
    with it; the MC will offer you a hard bargain, worse outcome, or
    tough choice.
\begin{movedetail}
  This is one of the most complex moves in the game, and also one of
  the most versatile. Any time you're trying to slip something by a
  person in conversation, or otherwise trick them, this is the move to
  go to. Any time you try to slip a subtle, harmonious magical
  effect past the filters of reality, this is the move to go
  to. Pretty much any magic that doesn't fit into one of the other
  moves probably can fit into this one.
\end{movedetail}
  \end{move}

  \vspace{1em}
  
  \SEPARATOR

  \begin{move}
    When you \TRIGGER{outfox, outpace, or outmaneuver} to avoid a
    danger, name your gambit and roll +Grace. On a 10+, you pull it
    off, just as described. On a 7-9 choose 1:
    \begin{itemize}
      \setlength\itemsep{0em}
      \item you overextend yourself in the process
      \item you come away with something missing
      \item someone else is drawn in
    \end{itemize}
\begin{movedetail}
  This move is particularly useful for when you're trying to do
  something quickly, or craftily. The MC might call for this move when
  you're trying to outpace a rogue mage in a foot-race, or when you're
  trying to avoid automagical countermeasures on that machine you just
  jacked yourself into. This move will save your bacon a hundred times
  over before you're done with it.
\end{movedetail}
  \end{move}

  \SEPARATOR

  \begin{move}
    When you \TRIGGER{keep your head down} to avoid attention, roll
    +Wits. On a 10+, you avoid notice, suspicion, or detection. On a
    7-9, someone on the alert notices you. Choose one:
    \begin{itemize}
      \setlength\itemsep{0em}
    \item Offer them something to ignore you
    \item Back off without arousing further suspicion
    \item Choose when and where you are noticed
    \end{itemize}
  \end{move}

  \SEPARATOR

  \begin{move}
    When you \TRIGGER{do Rending magic} and it affects a Lookout, gain
    2 Discord, Backlash, and roll +nothing (the Consensus doesn't care
    how smart or sexy you are).

    On a 10+ choose up to 3, on a 7-9 choose exactly 2.
    \begin{itemize}
      \setlength\itemsep{0em}
    \item You only gain 1 Discord
    \item You don't Backlash
    \item You don't cause collateral damage
    \item You don't add a Restriction to your Paradigm
    \end{itemize}
    On a miss, choose one, but not ``You don't Backlash.''
  \end{move}

  \SEPARATOR

  \begin{move}
    When you \TRIGGER{help or hinder} another PC, roll +Anchors. On a
    10+ they get a +1 or -2 to their roll, your choice. On a 7-9 they
    still get the bonus or penalty, but you open yourself to cost,
    danger, or retribution. If they are doing magic that is Aligned to
    your Paradigm, you may instead have them treat the result as one
    category higher or lower.
  \end{move}

  \columnbreak
  
  \begin{move}
    When you \TRIGGER{enact sudden, violent, or unexpected change on
      the physical world}, roll +Body. On a 10+, choose 3. On a 7-9,
    choose 2:
    \begin{itemize}
      \setlength\itemsep{0em}
    \item You accomplish what you set out to do
    \item You aren't injured in the process
    \item You don't cause any collateral damage
    \item You impress or dismay the target or an onlooker, and take +1
      forward interacting with them
    \end{itemize}
  \end{move}

  \SEPARATOR

  \begin{move}
    When you \TRIGGER{Backlash}, you pushed reality too far and it
    pushed back. Any effects besides Harm will be difficult if not
    impossible to fix with magic, particularly the type of magic that
    caused it. Roll +Discord, then reset Discord to 0.

    On a 10+, reality breaks around you a little bit. Make a Status
    Track for this break, and the MC will make one to track if you
    become convinced it is real. Decide its form: pervasive
    hallucinations, a fantasy world overlaying the real one, or a
    pocket dimension you fall into. Choose a Stressor motivating the
    break, and write a move to advance its Status Track.

    On a 7-9, choose 2:
    \begin{itemize}
      \setlength\itemsep{0em}
    \item You take 3 Harm
    \item You don't reset your Discord to 0
    \item A physical representation of your Discord manifests; the
      Consensus makes life noticeably inconvenient for you for the
      remainder of the session
    \item The Consensus manifests a being, an Enforcer of its will, to
      hunt you down, attack you, or otherwise warn you to change your
      behavior
    \end{itemize}

    On a miss, you shake it off with minimal lasting effects.
  \end{move}

\end{multicols}
\end{document}
